\documentclass[a4paper,11pt]{article}
\newcommand\tab[1][0.6cm]{\hspace*{#1}}
\usepackage[T1]{fontenc}
\usepackage[polish]{babel}
\usepackage[utf8]{inputenc}
\usepackage{lmodern}
\usepackage{hyperref}
\usepackage[top=2cm, bottom=2cm, left=2cm, right=2cm]{geometry}
\usepackage{listings}
\usepackage{amsmath}
\usepackage{graphicx}
\usepackage{float}

\title{ \sc{Specyfikacja implementacyjna} \\
\emph{Projekt indywidualny} }

\author{Łukasz Knigawka}

\begin{document}

\maketitle

\thispagestyle{empty}

\tableofcontents

\newpage

\section{Wstęp}
\tab W specyfikacji funkcjonalnej opisano kluczowe dla zrozumienia problemu pojęcia: \textit{wymiana waluty, kurs walutowy, opłata za wymianę, arbitraż}.
\\\tab W specyfikacji implementacyjnej nie zostały zdefiniowane terminy leżące u podstaw informatyki, takie jak \textit{graf, graf skierowany}. 
\\\tab Przy realizacji zadania wykorzystany zostanie język C\# w wersji 7.3, .NET framework w wersji 4.7.2. Rozwiązanie zostanie utworzone, przetestowane i uruchomione na komputerze z 64-bitowym systemem Windows10, procesorem Intel Core i7-6700HQ oraz pamięcią RAM 16GB. 



\section{Diagram klas}

\section{Opis algorytmu}

\end{document}