\documentclass[a4paper,12pt]{article}
\newcommand\tab[1][0.6cm]{\hspace*{#1}}
\usepackage[T1]{fontenc}
\usepackage[polish]{babel}
\usepackage[utf8]{inputenc}
\usepackage{lmodern}
\usepackage{hyperref}
\usepackage[top=2cm, bottom=2cm, left=2cm, right=2cm]{geometry}
\usepackage{listings}
\usepackage{amsmath}
\usepackage{graphicx}
\usepackage{float}

\title{ \sc{Specyfikacja implementacyjna} \\
\emph{Projekt indywidualny} }

\author{Łukasz Knigawka}

\begin{document}

\maketitle

\thispagestyle{empty}

\tableofcontents

\newpage

\section{Wstęp}
\tab W specyfikacji funkcjonalnej opisano kluczowe dla zrozumienia problemu pojęcia: \textit{wymiana waluty, kurs walutowy, opłata za wymianę, arbitraż}.
\\\tab W specyfikacji implementacyjnej nie zostały zdefiniowane terminy leżące u podstaw informatyki, takie jak \textit{graf, graf skierowany, algorytm, algorytm grafowy}. 
\\\tab W przedstawianej koncepcji implementacji rozwiązania problemu, opisane w \textit{specyfikacji funkcjonalnej} zadanie tłumaczy się na język informatyki poprzez przedstawienie problemu na grafie. Graf ten jest skierowany, przy czym krawędzie (skierowane) istnieją jedynie między tymi wierzchołkami, dla których takie połączenie wynikać będzie z danych wejściowych. Przedstawiana sytuacja wymiany walut tylko powierzchownie ma swoje odbicie na rzeczywistym rynku wymiany walut, na czas realizacji zadania zapominamy jednak o realiach współczesnego świata. 
\\\tab Przy realizacji zadania wykorzystany zostanie język C\# w wersji 7.3 oraz .NET framework w wersji 4.7.2. Interfejs graficzny zostanie utworzony dzięki bibliotece \textit{WindowsForms}, która jest przestarzała i nie pozwala na wykonywanie wizualnie zadowalających projektów, lecz pozwala na tworzenie bardzo szybkich prototypów. Taki wybór należy uzasadnić tym, iż sercem programu jest algorytm, a głównym celem zadania nie jest doskonalenie tworzenia aplikacji desktopowych. Rozwiązanie zostanie utworzone, przetestowane i uruchomione na komputerze z 64-bitowym systemem Windows10, procesorem Intel Core i7-6700HQ oraz pamięcią RAM 16GB. 

\section{Diagram klas}

\begin{figure}[H]
\centering
\includegraphics[width=15cm]a
\caption{Diagram klas}
\end{figure}

\tab Przedstawiony diagram klas ma na celu przybliżenie koncepcji rozwiązania problemu. Nie zawiera w sobie oczywistych pól i metod dostępowych. Nie ukazano wszystkich relacji między klasami. Jako że rozwiązanie nie zostało jeszcze zaimplementowane, niektóre rozwiązania zapewne zmienią swoją formę w ostatecznej wersji projektu.    

\section{Opis klas}
\subsection{Klasy MainForm oraz Program}
\tab Klasa \textit{Program} została pominięta na wykresie, z tego powodu że jest to typowa klasa dla architektury prostej aplikacji desktopowej. Uruchamiany jest za jej pomocą interfejs programu,  znajduje się w niej metoda \textit{Main}. \\
Klasa \textit{MainForm} zawiera metody obsługujące interakcję użytkownika z interfejsem graficznym. Obsłużony jest przycisk wyboru pliku, po którego kliknięciu pojawia się okno dialogowe wyboru pliku. Rozwiązanie jest utworzone w taki sposób, iż użytkownik nie jest w stanie wybrać pliku innego niż plik tekstowy o rozszerzeniu \textit{.txt}, gdyż tylko takie pliki są widoczne w eksploratorze. W klasie tej znajduje się też obsługa kliknięć pozostałych przycisków - odnalezienia korzystnej wymiany oraz odnalezienia arbitrażu. Pobierane są wtedy dane z wypełnionych pól tekstowych oraz opcjonalnie z pliku tekstowego, w zależności od wybranego rodzaju źródła danych wejściowych. Klasa komunikuje się z klasą przetwarzającą dane wejściowe i tworzącą graf. 

\subsection{Klasa Parser}
\tab Sercem tej klasy jest metoda \textit{Parse}, która przetwarza dane wejściowe. Bardzo możliwe, że aby zachować względną prostotę i czystość kodu konieczne będzie utworzenie mniejszych metod, które będą prowadziły do odczytania danych. Dane czytane są linijka po linijce - jeśli linia zaczyna się od znaku \textit{,,\#''}, to algorytm przechodzi do następnej linii. Jednocześnie, wraz z napotkaniem linii zaczynającej się od takiego znaku zwiększana jest zmienna lokalna świadcząca o spodziewanym typie linii. To znaczy, że po jednym odnotowanym wystąpieniu takiego znaku spodziewamy się linii opisujących waluty, a po dwóch wystąpieniach spodziewamy się opisu kursu walut. Dane są zbierane, a na ich podstawie budowane jest drzewo. Prawdopodobnie najtrudniejszym zadaniem jest obsłużenie błędnie sformatowanych danych.

\subsection{Klasy Graph, ExchangeEdge oraz CurrencyVertex}
\tab Klasa \textit{CurrencyVertex} symbolizuje wierzchołek grafu, który reprezentuje rodzaj waluty. Zawiera symbol waluty (trzyznakowy ciąg wielkich liter), informację czy wierzchołek był już odwiedzony (aby znaleźć korzystną wymianę walut), informację o najkrótszej ścieżce z wierzchołka źródłowego do danego wierzchołka, referencję do poprzednika oraz listę krawędzi tego wierzchołka. Zaleca się skorzystanie z \textit{właściwości zaimplementowanych automatycznie}. Klasa \textit{ExchangeEdge} symbolizuje wymianę walut. Oprócz wagi danej krawędzi należy zawrzeć w niej informację o rodzaju i wysokości towarzyszącej jej opłaty. Klasa \textit{Graph} zawiera w sobie zbiór wierzchołków i krawędzi. 

\subsection{Klasa Visionary}
\tab Klasa \textit{Visionary} zawiera w sobie algorytmy wyszukujące najkorzystniejszą wymianę oraz arbitraż. Z algorytmicznego punktu widzenia jest to najważniejsza klasa w projekcie.

\section{Opis algorytmu}
\tab Funkcje programu zostały opisane w \textit{specyfikacji funkcjonalnej}. Pierwsza z nich sprowadza się do znalezienia najkrótszej drogi z danego wierzchołka źródłowego do danego wierzchołka docelowego w grafie ważonym. Realizacja tego zadania będzie opierać się o odnalezienie najkrótszej ścieżki z wierzchołka źródłowego do pozostałych wierzchołków grafu. Druga z nich koncentruje się na wyszukaniu w grafie ujemnego cyklu. 
\\\tab W rozwiązaniu można wykorzystac algorytm \textit{Bellmana-Forda}. Jego zaletą względem algorytmu \textit{Dijkstry} w przypadku omawianego problemu jest możliwość występowania ujemnych wag wierzchołków. Gdybyśmy jako wierzchołki przyjęli waluty, a jako wagi krawędzi kursy walut, to nasz problem korzystnej wymiany sprowadzałby się do znalezienia najdłuższej ścieżki między walutą wymienianą i pożądaną. Natomiast jeśli wagi wierzchołków zlogarytmujemy i pomnożymy przez \textit{-1}, otrzymujemy problem znalezienia najkrótszej ścieżki w grafie. Do otrzymania logarytmu można wykorzystać metodę \textit{Math.Log()}, a chcąc dokonać powrotnej konwersji można wykorzystać \textit{Math.Exp()}.
\\\tab Realizacja algorytmu Bellmana-Forda przedstawiona w \textit{,,Wprowadzeniu do Algorytmów'' (Cormen, Rivest, Leiserson, Stein)} została przedstawiona poniżej.

\begin{figure}[H]
\centering
\includegraphics[width=7cm]b
\caption{Algorytm Bellmana-Forda zapisany w pseudokodzie}
\end{figure}

\tab W tym przypadku poprzez \textit{G} oznaczony jest graf, przez \textit{V} zbiór wierzchołków, przez \textit{E} zbiór krawędzi, przez \textit{w} funkcja wagowa, przez \textit{d} oszacowanie wagi najkrótszej ścieżki, przez \textit{s} źródło (\textit{source}), a przez \textit{u} i \textit{v} wierzchołki. Algorytm korzysta z \textit{relaksacji} krawędzi, czyli ze sprawdzenia, czy przechodząc przez wierzchołek \textit{u} można znaleźć krótszą od dotychczas najkrótszej ścieżki do \textit{v}. Jeśli to możliwe, aktualizowana jest wartość najkrótszej ścieżki do danego wierzchołka i aktualizowany jest poprzednik wierzchołka. Algorytm zwraca \textit{TRUE}, gdy graf nie zawiera cykli o ujemnych wagach (osiągalnych ze źródła). W przeciwnym wypadku zwraca \textit{FALSE}. Jest to więc perfekcyjny algorytm do wyznaczania arbitrażu, gdyż ten występuje gdy graf zawiera cykl ujemny.
\\\tab Utrudnieniem w realizacji zadania są dodatkowe opłaty za wykonanie każdej wymiany walutowej, jednak trzeba je po prostu uwzględnić przy szukaniu ścieżki w grafie, przy każdym porównaniu obecnej wagi drogi od źródła do konkretnego wierzchołka z alternatywą. 

\section{Testy}
\tab Zakłada się użycie przy tworzeniu testów jednostkowych \textit{Visual Studio Unit Testing Framework}. Narzędzie to znane jest także pod nazwą \textit{MSTest}. Oprócz testów jednostkowych zostanie przeprowadzone seria testów manualnych. Testy powinny według założeń powstawać w czasie pisania kodu programu. Większość testów jednostkowych będzie imitować testy manualne.

\end{document}