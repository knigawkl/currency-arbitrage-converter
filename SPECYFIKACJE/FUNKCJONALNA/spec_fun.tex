\documentclass[a4paper,11pt]{article}
\newcommand\tab[1][0.6cm]{\hspace*{#1}}
\usepackage[T1]{fontenc}
\usepackage[polish]{babel}
\usepackage[utf8]{inputenc}
\usepackage{lmodern}
\usepackage{hyperref}
\usepackage[top=2cm, bottom=2cm, left=2cm, right=2cm]{geometry}
\usepackage{listings}
\usepackage{amsmath}
\usepackage{graphicx}
\usepackage{float}

\title{ \sc{Specyfikacja funkcjonalna} \\
\emph{Projekt indywidualny} }

\author{Łukasz Knigawka}

\begin{document}



\maketitle

\thispagestyle{empty}

\tableofcontents

\newpage

\section{Wstęp teoretyczny}

\tab Wymianę waluty rozumiemy jako zamianę określonej kwoty w jednej walucie na kwotę w drugiej walucie. Wysokość kwoty waluty wynikowej jest zależna od kursu walutowego dla tych walut. Kurs walutowy jest ceną danej waluty wyrażoną w innej walucie. W opisywanym przypadku wymiana waluty nie odbywa się za darmo, a opłata za wymianę ma charakter procentowej wartości od wymienionej kwoty lub może być stałą opłatą manipulacyjną. W przypadku występowania opłaty stałej, jest ona zawsze wyrażana w walucie docelowej.
\\\tab Arbitraż rozumiemy dalej jako kombinację, w której kursy walut są tak ułożone, że wymiana waluty, w której waluta początkowa jest tożsama z walutą końcową, zwraca więcej niż kwota wejściowa.

\section{Funkcje programu}

\tab Program posiada dwie główne funkcje:
\begin{enumerate}
\item Wykrywanie korzystnej ścieżki wymiany waluty dla wskazanej waluty
\item Wykrywanie sytuacji ekonomicznego arbitrażu
\end{enumerate}

\section{Opis wejścia i wyjścia}

\begin{figure}[H]
\centering
\includegraphics[width=15cm] a
\caption{Fragment okna programu}
\label{fig:obrazek 1}
\end{figure}

\section{Sytuacje wyjątkowe}

\end{document}
