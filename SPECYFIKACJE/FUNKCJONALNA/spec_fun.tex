\documentclass[a4paper,11pt]{article}
\newcommand\tab[1][0.6cm]{\hspace*{#1}}
\usepackage[T1]{fontenc}
\usepackage[polish]{babel}
\usepackage[utf8]{inputenc}
\usepackage{lmodern}
\usepackage{hyperref}
\usepackage[top=2cm, bottom=2cm, left=2cm, right=2cm]{geometry}
\usepackage{listings}
\usepackage{amsmath}
\usepackage{graphicx}
\usepackage{float}

\title{ \sc{Specyfikacja funkcjonalna} \\
\emph{Projekt indywidualny} }

\author{Łukasz Knigawka}

\begin{document}

\maketitle

\thispagestyle{empty}

\tableofcontents

\newpage

\section{Wstęp teoretyczny}

\tab Wymianę waluty rozumiemy jako zamianę określonej kwoty w jednej walucie na kwotę w drugiej walucie. Wysokość kwoty waluty wynikowej jest zależna od kursu walutowego dla tych walut. Kurs walutowy jest ceną danej waluty wyrażoną w innej walucie. W opisywanym przypadku wymiana waluty nie odbywa się za darmo, a opłata za wymianę ma charakter procentowej wartości od wymienionej kwoty lub może być stałą opłatą manipulacyjną. W przypadku występowania opłaty stałej, jest ona zawsze wyrażana w walucie docelowej.
\\\tab Arbitraż rozumiemy dalej jako kombinację, w której kursy walut są tak ułożone, że wymiana waluty, w której waluta początkowa jest tożsama z walutą końcową, zwraca więcej niż kwota wejściowa.

\section{Funkcje programu}

\tab Program posiada dwie główne funkcje:
\begin{enumerate}
\item Wykrywanie korzystnej ścieżki wymiany waluty dla wskazanej waluty
\item Wykrywanie sytuacji ekonomicznego arbitrażu
\end{enumerate}
\tab Użytkownik otrzymuje powyższe główne funkcje dzięki poniższym funkcjom pomocniczym:
\begin{itemize}
\item Wybranie sposobu podania danych wejściowych 
\item Wpisanie danych wejściowych w pole tekstowe 
\item Podanie ścieżki do pliku z danymi wejściowymi
\item Wybranie oczekiwanego działania programu, to jest jednej funkcji głównej bądź obu
\item Wprowadzenie wielkości środków pieniężnych do korzystnej wymiany
\item Wprowadzenie waluty wyjściowej dla korzystnej wymiany walutowej
\item Wprowadzenie wielkości środków pieniężnych do wyszukania arbitrażu
\item Wypisanie wyników działania programu w polu wynikowym
\item Obsługa możliwych błędów oraz informowanie o ich rodzajach
\end{itemize}

\section{Opis wejścia i wyjścia}

\tab Komunikacja użytkownika z programem odbywa się przy pomocy interfejsu graficznego. Poniżej przedstawiono prototyp okna głównego programu.

\begin{figure}[H]
\centering
\includegraphics[width=15cm] a
\caption{Okno programu}
\label{fig:obrazek 1}
\end{figure}

\tab Użytkownik decyduje o sposobie podania danych wejściowych do programu. Do wyboru są dwie opcje, nie można wybrać obu równocześnie. Sposób formatowania pliku wejściowego tekstowego oraz danych wejściowych podanych poprzez okno programu nie różni się w żaden sposób. Przykładowe przedstawienie danych wejściowych:
 
\textit{
\\\tab \# Waluty (id | symbol | pełna nazwa)
\\\tab 0 EUR euro
\\\tab 1 GBP funt brytyjski
\\\tab 2 USD dolar amerykański
\\\tab \# Kursy walut (id | symbol (waluta wejściowa) | symbol (waluta wyjściowa) | kurs | typ opłaty | opłata
\\\tab 0 EUR GBP 0,8889 PROC 0,0001
\\\tab 1 GBP USD 1,2795 PROC 0
\\\tab 2 EUR USD 1,1370 STAŁA 0.025
\\\tab 3 USD EUR 0,8795 STAŁA 0.01
}
\\\\\tab W pierwszej kolejności należy podać występujące waluty. Tak jak zaprezentowano powyżej, należy podać identyfikator waluty, następnie jej skrót (trzyliterowy kod pisany wielkimi literami) oraz pełną nazwę.
\\Następnie należy podać możliwe wymiany walutowe. Po identyfikatorze wymiany należy podać skrót waluty wymienianej, skrót waluty będącej produktem wymiany, kurs wymiany, rodzaj opłaty za wymianę oraz jej wartość liczbową. Możliwe rodzaje opłaty za wymianę to \textit{PROC} oraz \textit{STAŁA}. Przed podaniem występujących walut oraz przed podaniem wymian należy umieścić dokładnie taki schemat opisu jak na \textit{Rysunku 1}. Nie należy w pliku wejściowym umieszczać jakichkolwiek innych informacji.
\\\tab W przypadku wybrania opcji podawania danych wejściowych przy pomocy pliku tekstowego, to jest po kliknięciu przycisku \textit{Open file}, pojawia się okno dialogowe, umożliwiające wybór pliku. Nie ma możliwości wyboru pliku innego niż plik tekstowy. Użytkownik nie zobaczy w wyświetlonym oknie-eksploratorze plików innych niż pliki tekstowe, co przedstawiono na \textit{Rysunku 2}.  

\begin{figure}[H]
\centering
\includegraphics[width=15cm] b
\caption{Okno dialogowe służące załadowaniu pliku wejściowego}
\label{fig:obrazek 1}
\end{figure}

\tab Na \textit{Rysunku 3} przedstawiono okno główne programu po wybraniu pliku wejściowego tekstowego. W polu tekstowym pojawiła się ścieżka do tego pliku.

\begin{figure}[H]
\centering
\includegraphics[width=15cm] c
\caption{Okno programu po wybraniu pliku wejściowego}
\label{fig:obrazek 1}
\end{figure}

\tab Użytkownik proszony jest także o wybór oczekiwanego rodzaju wyniku. Wynik ten pojawi się w nieedytowalnym polu tekstowym umieszczonym na dole okna głównego programu. Niezbędne jest także podanie wartości pieniężnej waluty wymienianej i rodzaju waluty wyjściowej w przypadku poszukiwania korzystnej ścieżki wymiany, albo wartości pieniężnej dla której poszukujemy arbitrażu.\\
Przykładowy wynik działania programu w przypadku wybrania trybu najkorzystniejszej wymiany waluty:\\\\
\textit{1000 EUR -> GBP -> USD -> 1137,34755 USD}\\\\
Wynik ten zawiera wartość pieniężną początkową wymienianą i jej walutę, następnie szereg walut pośrednich, a także wartość pieniężną końcową oraz jej walutę. \\Przykładowy wynik działania programu w przypadku wybrania trybu wyszukania dowolnego arbitrażu:\\\\
\textit{1000 EUR -> GBP -> USD -> EUR -> 1000,297170225 EUR}\\\\
W takim przypadku wynik zawiera wartość pieniężną początkową i jej walutę, następnie szereg walut pośrednich, a także wartość pieniężną końcową przedstawioną we wcześniej przedstawionej walucie początkowej.

\section{Sytuacje wyjątkowe}

\tab Częścią programu jest obsługa sytuacji wyjątkowych i błędów. Poniżej znajduje się lista problemów, jakie może napotkać program oraz jego reakcje. Sytuacje te podzielić można na kilka grup:
\\\begin{enumerate}
\item Sytuacje związane z danymi wejściowymi
\begin{itemize}
\item Gdy zabraknie schematu opisu, wyświetlone zostanie okno dialogowe z informacją o braku schematu opisu
\item Gdy zabraknie \textit{identyfikatora waluty/symbolu waluty/pełnej nazwy waluty}, wyświetlone zostanie okno dialogowe z informacją o braku \textit{identyfikatora waluty/symbolu waluty/pełnej nazwy waluty} 
\item Gdy zabraknie \textit{identyfikatora kursu, symbolu waluty wejściowej, symbolu waluty wyjściowej, kursu, typu opłaty lub opłaty}, wyświetlone zostanie okno dialogowe z informacją o braku \textit{identyfikatora kursu, symbolu waluty wejściowej, symbolu waluty wyjściowej, kursu, typu opłaty lub opłaty}

\end{itemize}
\item Sytuacje związane z plikiem wejściowym tekstowym
\begin{itemize}
\item Gdy nie uda się otworzyć pliku wejściowego, pojawi się okno dialogowe z komunikatem \textit{Could not open file!}
\item Gdy plik wejściowy okaże się pusty, pojawi się okno dialogowe z komunikatem \textit{Could not read an empty file!}
\end{itemize}
\item Sytuacje związane z niemożnością odnalezienia satysfakcjonującego wyniku programu
\begin{itemize}
\item W przypadku gdy nie uda się znaleźć korzystnego sposobu wymiany waluty, w wynikowe pole tekstowe zostanie wpisane: \textit{Currency exchange: Could not find any beneficial exchange!}
\item W przypadku gdy nie uda się odnaleźć arbitrażu, w wynikowe pole tekstowe zostanie wpisane \textit{Arbitrage finding: Could not find any arbitrage!}
\end{itemize}
\end{enumerate}


\end{document}
